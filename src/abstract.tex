\addcontentsline{toc}{chapter}{ABSTRACT}
\chapter*{ABSTRACT}

One of the defining factors in deciding the success of a Search and Rescue (SAR) operation during a natural disaster is in how quickly victims can be found and delivering life-sustaining logistics for said victim. The usage of Unmanned Aerial Vehicles can support SAR operations, and to improve the coordination between UAVs, an inter-UAV communications algorithm is developed. The type of communication system developed is a Flying Ad-hoc Network (FANET), and the communications technology chosen to build the FANET is Wi-Fi using ESP32 microcontrollers and the PainlessMesh library. This research built a mesh network of ESP32 nodes, with each node being placed up to 50 meters apart, in which it is able to send GPS location data from the sender drone to the receiver drone with a packet loss of x percent, median throughput of x B/s, and median round-trip delay of x ms. Based on linear regression analysis, there is a correlation between throughput, round-trip delay, and packet loss values to the signal strength between nodes, with a lower signal strength between nodes resulting in lowered network performance.

\vspace{2em}
\noindent \textbf{Keywords:} PainlessMesh, ESP32, Unmanned Aerial Vehicles, Drone, Communications