\addcontentsline{toc}{chapter}{ABSTRAK}
\chapter*{ABSTRAK}

Salah satu faktor penentu kesuksesan dalam operasi \textit{Search and Rescue} (SAR) pada bencana alam adalah kecepatan dalam menemukan lokasi dan posisi korban serta pengiriman logistik bantuan penunjang hidup bagi korban tersebut. Penggunaan \textit{Unmanned Aerial Vehicle} dapat mendukung operasi SAR, dan untuk meningkatkan koordinasi antar UAV dalam sistem maka dikembangkan sebuah algoritma komunikasi antar-UAV. Model komunikasi yang dikembangkan berupa \textit{Flying Ad-hoc Network} (FANET), dan teknologi komunikasi yang dipilih berupa Wi-Fi menggunakan mikrokontroler ESP32 dan \textit{library} PainlessMesh. Kinerja jaringan diuji dari nilai \textit{throughput, packet loss,} dan \textit{round-trip delay}. Penelitian ini menghasilkan sebuah jaringan mesh dengan jarak antar node ESP32 hingga 50 meter, mampu mengirimkan data lokasi dari drone \textit{sender} ke drone \textit{receiver} dengan nilai \textit{packet loss} sebesar x persen, \textit{throughput} jaringan median sebesar x B/s, dan \textit{round-trip delay} median sebesar x ms. Analisis regresi linear menunjukkan korelasi antara \textit{throughput, round-trip delay,} dan \textit{packet loss} terhadap \textit{signal strength} antar node, dimana nilai \textit{signal strength} yang semakin mengecil menghasilkan kinerja jaringan yang menurun.

\vspace{2em}
\noindent \textbf{Kata kunci:} PainlessMesh, ESP32, \textit{Unmanned Aerial Vehicles}, \textit{Drone}, Komunikasi